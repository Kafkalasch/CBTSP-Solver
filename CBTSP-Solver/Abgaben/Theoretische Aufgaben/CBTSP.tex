\documentclass[10pt,oneside,a4paper]{report}
\usepackage[utf8]{inputenc}
\usepackage{amsmath}
\usepackage{amsfonts}
\usepackage{amssymb}
\usepackage{amsthm}
\usepackage{mathtools}
\usepackage[ngerman]{babel}

\newcommand{\nat}{\mathbb{N}}
\newcommand{\pot}[1]{\mathfrak{P}(#1)} 
\newcommand{\norm}[1]{\left\|#1\right\|}
\newtheorem{Def}{Definition}
\newtheorem{Satz}[Def]{Satz}
\newtheorem{Bem}[Def]{Bemerkung}
\newtheorem{Bsp}[Def]{Beispiel}
\newtheorem{Fol}[Def]{Folgerung}
\newtheorem{Lem}[Def]{Lemma}
\newtheorem{Prop}[Def]{Proposition}



\begin{document}
	\begin{center}
		\textbf{CBTSP - Projekt}
		
		\textit{Michael Höfler, Claudius Röhl}
	\end{center}

\begin{Def}  
	Sei $K$ eine Menge, $E\subseteq K\times K$ eine Relation über $K$ und $C:E\to\mathbb{R}$.  Wir nennen das Tupel $G:=(K,E,C)$ einen \textbf{Graphen}, die Elemente aus $K$ \textbf{Knoten}, die aus $E$ \textbf{Kanten} und $C$ die \textbf{Kostenfunktion} von $G$. \\
	
	\noindent Wenn $\forall (a,b)\in E: (b,a)\in E$ gilt ($E$ also symmetrisch ist), nennen wir $G$  einen \textbf{ungerichteten Graphen}. In diesem Fall können die Kanten auch als Teilmenge $\tilde{E}$ der Potenzmenge von $K$ geschrieben werden, indem wir 
	$$\tilde{E}:=\{\{a,b\}\in \pot{K}\mid a,b\in K: (a,b)\in E\}$$
	setzen und $(K,E)$ mit $(K,\tilde{E})$ identifizieren.
\end{Def}

\begin{Def} 
	Sei $G=(K,E,C)$ ein Graph und $K'\subseteq K$, $E'\subseteq (K'\times K')\cap E$ sowie $C|_{G'}=C'$. Dann heiße $G':=(K',E',C')$ ein \textbf{Teilgraph} von $G$ und wir schreiben $G'\subseteq G$.
\end{Def}

\begin{Def} 
	Sei $G=(K,E,C)$ ein ungerichteter Graph. Eine Folge 
	$$H=(h_1,h_2...,h_{|G|}), h_i \in K$$
	heiße \textbf{hamiltonscher Zyklus}, falls
	$$\{ h_i \mid i\in\nat_{1,|G|}\} = G$$ und
	für alle $1\leq i<j\leq |G|$ gilt
	$$ h_i\neq k_j \wedge \{h_i,h_j\}\in E.$$
	Wir nennen
	$$C_{CBTSP}(H):=\left|\sum_{i = 1}^{|G|-1}C(\{h_i,h_{i+1}\})\right|\in \mathbb{R}$$
	die CBTSP-Kosten des Zyklus H.
	Wir setzen 
	$$E_H := \{h\in E\mid \exists n\in \nat: h=(h_n,h_{n+1})\}.$$
\end{Def}

\begin{Satz}
	Seien $G=(K,E,C)$ ein endlicher ungerichteter Graph und $G'\subsetneq G$ ein echter Teilgraph dergestalt, dass
	$$C(x) =
	\begin{cases}
		C'(e) & \text{falls } e\in E',\\
		2M+1 & sonst
	\end{cases}$$
	mit 
	$$M:= \sum_{e'\in E'}|C'(e)|.$$
	Seien $H$ und $H'$ hamiltonsche Zyklen von $G$ bzw. $G'$, wobei $H$ kein hamiltonscher Zyklus von $G'$ sei.
	Dann gilt:
	$$C_{CBTSP}(H') < C_{CBTSP}(H).$$
	\begin{proof}
		Sei alles wie oben. Da $H=(h_1,...,h_{|G|})$ kein hamiltonscher Zyklus von $G'$ ist, gibt es ein $s\in \nat$ so, dass $\hat{h}:=(h_s,h_{s+1})\notin E'$. Daher gilt:
		$$C(\hat{h}) = 2M+1.$$
		Weiters haben wir
		\begin{eqnarray}
		C_{CBTSP}(H')&=&\left|\sum_{h'\in {E'}_{H'}}C(h')\right| \\
		&\leq&\sum_{h'\in {E'}_{H'}}|C(h')|\\
		&\leq& \sum_{h'\in E'}|C(h)| \\
		&=& M\\
		&<& M + 1\\
		&\leq& \left|\underbrace{\left(\sum_{h\in {E_H\backslash\{\hat{h}\}}}C(h)\right) + M}_{\geq 0} + M +1\right|\\
		&=& \left|\sum_{h\in {E}_{H}}C(h)\right|\\
		&=& C_{CBTSP}(H).
		\end{eqnarray}
	\end{proof}
\end{Satz}


%\begin{Def} 
%	Sei $G=(K,E)$ ein Graph. $G$ heiße zusammenhängend, wenn es für je zwei Knoten $a,b\in K$ einen Weg von $a$ nach $b$ gibt.
%	%Die Menge aller zusammenhängenden Teilgraphen von $G$ wird durch $\mathcal{Z}(G)$ bezeichnet.
%\end{Def}




\end{document}