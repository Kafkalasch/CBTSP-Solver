\documentclass[10pt,twoside,a4paper]{book}
\usepackage[utf8]{inputenc}
\usepackage{amsmath}
\usepackage{amsfonts}
\usepackage{amssymb}
\usepackage{amsthm}
\usepackage{mathtools}
\usepackage[ngerman]{babel}
\usepackage{marginnote}
\usepackage{enumerate}
\usepackage{dsfont}
\usepackage{mathrsfs}
\usepackage{tikz}
\usetikzlibrary{arrows,automata,positioning,intersections,calc,through,backgrounds,patterns}



%Einstellungen der Seitenr�nder
\usepackage[left=2.5cm,right=3cm,top=2cm,bottom=3cm,includeheadfoot]{geometry} %left left=2.5cm,right=3cm,top=2cm,bottom=3cm,includeheadfoot]{geometry
\usepackage{setspace}
\setstretch{0,97}%{1,15}
\usepackage{makeidx}
\makeindex



%\renewcommand{\theequation}{\arabic{section}.\arabic{equation}} 
%\renewcommand{\thesection}{\arabic{section}\setcounter{equation}{0}} 
%\usepackage{../../demo2}
% selbstdefinierte Makros, ... aus demo2.sty
\newcommand{\iindex}[1]{\index{#1}#1}
\newcommand{\bigslant}[2]{{\raisebox{.2em}{$#1$}\left/\raisebox{-.2em}{$#2$}\right.}}
\newcommand{\re}{\text{Re}\hspace{0.1 cm}}
\newcommand{\im}{\text{Im}\hspace{0.1 cm}}
\newcommand{\RM}[1]{\MakeUppercase{\romannumeral #1{.}}} 
\newcommand{\Ord}{\mathfrak{Ord}}
\newcommand{\integral}{\int\limits}
\newcommand{\nat}{\mathbb{N}}
\newcommand{\reell}{\mathbb{R}}
\newcommand{\ganz}{\mathbb{Z}}
\newcommand{\komplex}{\mathbb{C}}
\newcommand{\rat}{\mathbb{Q}}
\newcommand{\durchschnitt}{\mathop{\bigcap}\limits}
%\newcommand{\Vereinigung}{\mathop{\bigcup}\limits}
\newcommand{\reellzwei}{\reell^{\,2}}
\newcommand{\gleichdrueber}[1]{\stackrel{(#1)}{=}}
\newcommand{\kgleichdrueber}[1]{\stackrel{(#1)}{\leq}}
\newcommand{\bem}{\textit{Bemerkung: }}
%\newcommand{\Def}{\subsubsection*{Definition}}
%\newcommand{\Bsp}{\subsubsection*{Beispiel}}
%\newcommand{\Satz}{\subsubsection*{Satz}}
%\newcommand{\Beweis}{\subsubsection*{Beweis}}
\newcommand{\matrixeinfach}[1]{\begin{pmatrix}
#1_1 \\ #1_2 \\ \vdots \\ #1_n \end{pmatrix}}
\newcommand{\mxdots}[2]{\begin{pmatrix}
#1 \\ \dots \\ #2 \end{pmatrix}}
\newcommand{\Lemma}{\subsubsection{Lemma}}
\newcommand{\Korollar}{\subsubsection{Korollar}}
\newcommand{\Theorem}{\subsubsection{Theorem}}
\newcommand{\ringschluss}[1]{\begin{enumerate}[(i) $\Rightarrow$  
\ifnum \value{enumi}=#1 
\setcounter{enumi}{0}
\fi \addtocounter{enumi}{1} (i): \addtocounter{enumi}{-1}]}
% \begin{enumerate}[(i) $\Rightarrow$  \ifnum \value{enumii}=3 \setcounter{enumii}{0} \fi \addtocounter{enumii}{1} (i): \addtocounter{enumii}{-1}] \end{enumerate} 
\newcommand{\ringschlusszwei}[1]{\begin{enumerate}[(i) $\Rightarrow$  \ifnum \value{enumii}=#1 \setcounter{enumii}{0} \fi \addtocounter{enumii}{1} (i): \addtocounter{enumii}{-1}]}
\newcommand{\ringschlussanders}[1]{\begin{enumerate}[(i) $\Leftarrow$  
\ifnum \value{enumi}=#1 
\setcounter{enumi}{0}
\fi \addtocounter{enumi}{1} (i): \addtocounter{enumi}{-1}]}
\newcommand{\summe}{\sum\limits}
\newcommand{\Rightarrows}[1]{\stackrel{#1}{\Rightarrow}}
\newcommand{\natBasis}{e_1^{(q)}, \dots , e_q^{(q)}}
\newcommand{\reelln}{\reell^{n}}
\newcommand{\fracpartial}[2]{\frac{\partial #1}{\partial #2}}
\newcommand{\limes}[1]{\lim\limits_{#1}}
\newcommand{\Limes}[2]{\lim\limits_{#1 \rightarrow #2}}
\newcommand{\limesn}{\limes{n \rightarrow \infty}}
\newcommand{\fracd}[2]{\frac{\mathrm{d}#1}{\mathrm{d}#2}}
\newcommand{\nullover}[1]{\mathop{#1}\limits^\circ}
\newcommand{\reellN}{\reell^{\;N}}
\newcommand{\tillinfty}[1]{_{#1}^\infty}
\newcommand{\till}[2]{_{#1}^{#2}}
\newcommand{\bdots}[2]{#1_1 , \dots , #1_{#2}}
\newcommand{\pot}[1]{\mathcal{P}(#1)} 
\newcommand{\zfc}{\mathcal{ZFC} }
\newcommand{\zf}{\mathcal{ZF} }
\newcommand{\AC}{\mathcal{AC} }
\newcommand{\Kard}{\mathfrak{Kard}}
\newcommand{\kardprod}{\scalebox{0.7}{$\mathfrak{K}$}\hspace{-1em}\prod}
\DeclareMathOperator{\Grad}{Grad}
\DeclareMathOperator{\ann}{Ann}
\DeclareMathOperator{\End}{End}
\DeclareMathOperator{\Bild}{Bild}
\DeclareMathOperator{\zereq}{\, \stackrel{\hspace{-0.01 cm}\scalebox{0.5}{$\mathcal{Z}$}}{\sim}\, }
\newcounter{serie}
\newcommand{\serie}{\addtocounter{serie}{1}Serie \arabic{serie}}
\newcommand{\card}{\text{card }}
\newcommand{\vektor}[2]{\left(\begin{array}
{#1} #2\end{array}\right)}
\newcommand{\produkt}{\prod\limits}
\newcommand{\ninnat}{n \in \nat}
\newcommand{\norm}[1]{\left\|#1\right\|}
\newcommand{\dom}[1]{\text{dom}\left(#1\right)}
\DeclareMathOperator{\sign}{sign}
\DeclareMathOperator{\diver}{div}
\DeclareMathOperator{\rot}{rot}
\DeclareMathOperator{\spur}{spur}
\DeclareMathOperator{\supp}{supp}
\DeclareMathOperator{\rank}{rank}
\DeclareMathOperator{\diag}{diag}
\DeclareMathOperator{\bild}{bild}
\DeclareMathOperator{\kernn}{kern} 
\DeclareMathOperator{\cokern}{cokern}
\DeclareMathOperator{\cobild}{cobild}
\DeclareMathOperator{\rang}{rang}
\DeclareMathOperator{\esssup}{ess \; sup}
\DeclareMathOperator{\tr}{tr}
\DeclareMathOperator{\indexprod}{\makebox[12 pt][c]{$\overset{\hspace{0pt}\scalebox{1.4}{.}}{\times}$}}
\newcommand{\skalar}[1]{\left\langle #1 \right\rangle}
\newenvironment{aebeweis}[1]
{
%\newcounter{aezaehler}
\setcounter{aezaehler}{1}
\begin{list}{(\roman{aezaehler}) $\Rightarrow$  \ifnum \value{aezaehler}=#1 \setcounter{aezaehler}{0} \fi \addtocounter{aezaehler}{1} (\roman{aezaehler}) $\colon$}{\setlength{\labelwidth}{1em}}
}{\end{list}}
\DeclareMathOperator{\cov}{Cov}
\DeclareMathOperator{\spn}{span}
\DeclareMathOperator{\si}{Si}
\DeclareMathOperator{\diam}{diam}
\DeclareMathOperator{\ind}{ind}
\DeclareMathOperator{\imteil}{Im}
\DeclareMathOperator{\reteil}{Re}
\DeclareMathOperator{\grad}{grad}
\DeclareMathOperator{\dist}{dist}
\DeclareMathOperator{\eps}{eps}
\newcommand{\Cov}{\boldsymbol{\operatorname{Cov}}}
\newcommand{\Var}{\boldsymbol{\operatorname{Var}}}
\newcommand{\Pmass}{\boldsymbol{\operatorname{P}}}
\newcommand{\E}{\boldsymbol{\operatorname{E}}}
\newcommand{\source}[1]{% %für schöne Quellenangabe bei Zitaten
	\nobreak\parbox[t]{\linewidth}{\raggedleft #1}% Placing a quote source
}%
\newcommand{\explain}[2]{\underset{\mathclap{\overset{\uparrow}{#2}}}{#1}}
\newcommand{\explainup}[2]{\overset{\mathclap{\underset{\downarrow}{#2}}}{#1}}
\newcommand{\leer}{\emptyset}
\newtheoremstyle{dotless}{}{}{}{}{\bfseries}{}{5 pt}{}
\theoremstyle{dotless}

\newtheorem{Def}{Definition}[section]
\newtheorem{Satz}[Def]{Satz}
\newtheorem{Bem}[Def]{Bemerkung}
\newtheorem{Bsp}[Def]{Beispiel}
\newtheorem{Fol}[Def]{Folgerung}
\newtheorem{Lem}[Def]{Lemma}
\newtheorem{Prop}[Def]{Proposition}

%
\usepackage[pdftex, plainpages=false,bookmarks,bookmarksopen,colorlinks=true,linkcolor=black,citecolor=black,urlcolor=black]{hyperref}
